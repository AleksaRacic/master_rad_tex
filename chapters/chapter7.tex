\chapter{Закључак}

У овом раду представљен је систем финансијског агента који користи велики језички модел (LLM) као судију за евалуацију својих одговора на финансијска питања. Главни циљ рада био је да се испита ефикасност овог приступа у побољшању квалитета одговора кроз прилагођавање промпта и ефикасност коришћења LLM-а као судије.
\newline

Резултати су показали значајно побољшање у тачности одговора када је коришћен прилагођени промпт у поређењу са основним моделом. Међутим није јасно да ли је агент показао могућност резоновања или је само боље проналазио информације у контексту. Поготово ако узмемо у обзир друге метрике где није било значајне разлике. Потребно је урадити дубљу анализу како би се утврдило да ли је побољшање последица бољег резоновања или само боље претраге информација. У будућности је потребно осмислити методологију која би подигла ниво тачности на виши ниво. Такође је потребна анализа колико су изабрани алати учинковити и да ли би избор других алата могао донети боље резултате.
\newline

Што се тиче коришћења LLM-а као судије, резултати основног модела су конзистентни са тачношћу  у истраживању које је урађено на оргиналном FinanceBench скупу података \cite{yang_evaluating_2025}, што указује на потенцијал овог приступа. За остале критеријуме је тешко одредити ефикасност овог приступа јер метрике могу бити субјективне. Потребно је урадити додатни преглед како би се боље разумело колико је LLM као судија поуздан у процени квалитета одговора. Генерално, упитно је колико се судија ослања на референтни одговор, а колико на своје знање и да ли је референтни одговор једини прави начин да се одговори на дато питање.
\newline



