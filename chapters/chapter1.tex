\chapter{Увод}
\label{sec:1}

\section{Значај финансијских текстуалних података}

Савремени финансијски екосистем генерише огромну количину текстуалних података — од вести у реалном времену и аналитичких извештаја до регулаторних пријава и објава на друштвеним мрежама. Ови неструктурирани текстови носе кључне увиде о тржишним условима и основама пословања компанија, који често утичу на перцепцију и одлуке инвеститора. 

Процењује се да најмање 80\% свих данашњих података чине неструктурирани подаци, што обухвата и наведене финансијске текстуалне токове \cite{rocha_discovering_2021}. Суочавање са овим таласом информација постало је озбиљан изазов: огроман број дневних финансијских вести и извештаја једноставно је немогуће да било који човек у целости прочита и обради. 

Важни детаљи лако могу промаћи када су аналитичари затрпани документима од стотину страница или непрекидним током вести. Због тога је неопходно развијати алате и технике који помажу разумевању и извлачењу увида из великих количина финансијског текста. 

Ефикасна анализа ових текстуалних извора критична је не само за инвеститоре и аналитичаре који желе да доносе информисане одлуке, већ и за регулаторе и истраживаче који се ослањају на квалитативне информације које сирови квантитативни подаци не могу да обухвате.

\section{НЛП и велики језички модели мењају анализу финансија}

Напредак у машинском учењу, а посебно у обради природног језика (НЛП), драматично је променио начин на који финансијска индустрија обрађује текстуалне податке. 

НЛП омогућава рачунарима да тумаче и уносе структуру у неструктуриран текст, претварајући квалитативне информације у квантитативне сигнале или сажетке који су знатно лакши за анализу. 

На пример, задаци који би за човека били изузетно временски захтевни — преглед хиљада новинских чланака ради процене сентимента, читање транскрипата позива поводом зарада ради идентификације кључних тема, или поређење језика у више годишњих извештаја — сада се могу обавити у релативно кратком времену. 

НЛП алати могу брзо да обраде масивне количине текста како би уочили трендове, измерили сентимент и истакли потенцијалне ризике, суштински претварајући људски језик у примењива сазнања \cite{paro_ai_strategic_2023}. 

Кроз рударење и анализу великих текстуалних корпуса, савремени НЛП системи помажу у подршци пословним процесима, откривању макроекономских сигнала и побољшању доношења одлука у финансијским институцијама \cite{yang_evaluating_2025}.

Велики скок у способностима НЛП-а донели су велики језички модели (LLM). То су дубоки модели тренирани на огромним корпусима текста, који омогућавају висок ниво разумевања језика и генерисања текста. 

Модели попут GPT-4 показали су изузетну дубину разумевања, у стању да интерпретирају нијансе финансијског језика и контекста након обуке на великим скуповима текстова \cite{paro_ai_strategic_2023}. 

За разлику од ранијих НЛП система који су често захтевали специфично дотренирање за сваки задатак, LLM-ови могу да решавају широк распон задатака уз минимално или без додатног тренирања, захваљујући општем језичком и светском знању које су усвојили. 

У финансијском домену ови модели могу да произведу организоване, експертске анализе сложених докумената, идући даље од површинске читљивости ка разумевању суптилних детаља и доменске терминологије \cite{yang_evaluating_2025}. 

Један упит LLM-у може да да сажетак годишњег извештаја од 100 страница или одговори на детаљна питања о његовом садржају, практично симулирајући рад искусног финансијског аналитичара.

\section{Улога 10-K извештаја као кључних финансијских докумената}

Међу бројним текстуалним изворима у финансијама, годишњи 10-K извештаји које подносе америчке јавне компаnije издвајају се као посебно важни и информативни. Комисија за хартије од вредности САД (SEC) налаже да компаније поднесу 10-K, свеобухватне годишње извештаје који дају детаљан приказ финансијских резултата и пословних активности фирме \cite{yang_evaluating_2025}. Типичан 10-K садржи ревидиране финансијске извештаје, уз опширне наративне сегменте у којима се разматрају пословни модел и стратегија, тржишни услови, фактори ризика. У суштини, он пружа холистичну и структуирану слику рада и планова компаније. Кључно је да ова документа укључују и квалитативне информације — попут објашњења стратегије и процена ризика — које квантитативне метрике саме по себи не могу да ухвате.

Ипак, 10-K извештаји су озлоглашено обимни и сложени, често имајући стотине страница густог текста, правне терминологије и индустријског жаргона. То их чини тешким за тумачење \cite{yang_evaluating_2025}. Неструктурисана природа наратива отежава проналажење конкретних информација или упоредивост између компанија. Применом техника НЛП-а на 10-K извештајима могу се аутоматски извући корисни увиди и обрасци који би другачије тешко били уочени \cite{yang_evaluating_2025}.

\section{Циљ рада}
Као одговор на наведене изазове и могућности које пружају савремени NLP алати, овај рад је усмерен на развој и евалуацију финансијског агента заснованог на великом језичком моделу (LLM). Конкретно, користимо LLM опште намене уз конструисање контекста како бисмо омогућили агенту да обрађује сложене финансијске документе. Поред самог агента, уводимо и иновативан приступ процене његове успешности: други LLM служи као судија који оцењује и анализира одговоре агента по критеријумима тачности и квалитета. 