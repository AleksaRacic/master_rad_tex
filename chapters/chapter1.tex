\chapter{Увод}
\label{sec:1}

Савремени финансијски екосистем генерише огромну количину текстуалних података, од вести у реалном времену и аналитичких извештаја до регулаторних пријава и објава на друштвеним мрежама. Ови неструктурирани текстови носе кључне увиде о тржишним условима и основама пословања компанија, који често утичу на перцепцију и одлуке инвеститора. Процењује се да најмање 80\% свих данашњих података чине неструктурирани подаци, што обухвата и наведене финансијске текстуалне токове \cite{rocha_discovering_2021}. Суочавање са овом количином информација постало је озбиљан изазов: огроман број дневних финансијских вести и извештаја једноставно није могуће да било који човек у целости прочита и обради. 
\newline

Важни детаљи лако могу промаћи када су аналитичари затрпани документима од стотину страница или непрекидним током вести. Због тога је неопходно развијати алате и технике који помажу разумевању и извлачењу увида из великих количина финансијског текста. Ефикасна анализа ових текстуалних извора критична је не само за инвеститоре и аналитичаре који желе да доносе информисане одлуке, већ и за регулаторе и истраживаче који се ослањају на квалитативне информације које сирови квантитативни подаци не могу да обухвате.

Напредак у машинском учењу, а посебно у обради природног језика (енг. \textit{Natural Language Processing - NLP}), драматично је променио начин на који финансијска индустрија обрађује текстуалне податке. Ова обрада омогућава рачунарима да тумаче и уносе структуру у неструктуриран текст, претварајући квалитативне информације у квантитативне сигнале или сажетке који су знатно лакши за анализу. 
\newline

На пример, задаци који би за човека били изузетно временски захтевни — преглед хиљада новинских чланака ради процене сентимента, читање транскрипата позива поводом зарада ради идентификације кључних тема, или поређење језика у више годишњих извештаја — сада се могу обавити у релативно кратком времену. Алати обраде природног језика могу брзо да обраде велике количине текста како би уочили трендове, измерили сентимент и истакли потенцијалне ризике, суштински претварајући људски језик у примењива сазнања \cite{paro_ai_strategic_2023}. Кроз анализу великих текстуалних корпуса, савремени системи помажу у подршци пословним процесима, откривању макроекономских сигнала и побољшању доношења одлука у финансијским институцијама \cite{yang_evaluating_2025}.
\newline

Велики скок у способностима обраде природног језика донели су велики језички модели - ВЈМ ( енг. \textit{Large Language models - LLM} ). То су дубоки модели тренирани на огромним корпусима текста, који омогућавају висок ниво разумевања језика и генерисања текста. Модели попут GPT-4 показали су изузетну дубину разумевања, у стању да интерпретирају нијансе финансијског језика и контекста након обуке на великим скуповима текстова \cite{paro_ai_strategic_2023}. 
\newline

За разлику од ранијих система који су често захтевали специфично дотренирање за сваки задатак, велики језички модели могу да решавају широк распон задатака уз минимално или без додатног тренирања, захваљујући општем језичком и светском знању које су усвојили. У финансијском домену ови модели могу да произведу организоване, експертске анализе сложених докумената, идући даље од површинске читљивости ка разумевању суптилних детаља и доменске терминологије \cite{yang_evaluating_2025}. Један упит LLM-у може да да сажетак годишњег извештаја од 100 страница или одговори на детаљна питања о његовом садржају, практично симулирајући рад искусног финансијског аналитичара.

Као одговор на наведене изазове и могућности које пружају савремени NLP алати, овај рад је усмерен на развој и евалуацију финансијског агента заснованог на великом језичком моделу. Конкретно, користи се велики језички модел опште намене уз конструисање контекста како би се омогућило агенту да обрађује сложене финансијске документе. Поред самог агента, уводен и иновативан приступ процене његове успешности: други LLM служи као судија који оцењује и анализира одговоре агента по критеријумима тачности и квалитета. 
\newline

Такође ће се разматрати и значај проширеног контекста за сам квалитет одговора. Уз помоћ агента-судије, биће систематски анализирано у којој мери проширење контекстуалних информација утиче на тачност, потпуност и квалитет одговора на сложена финансијска питања. На овај начин, добија се увид не само у ефикасност предложеног финансијског агента, већ и у општу применљивост техника рада са великим језичким моделима у специјализованом домену.