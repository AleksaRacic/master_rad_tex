\chapter{Подаци}
\label{sec:3}

\section{Преглед базе података EDGAR}

Програм електронског прикупљања, анализе и повлачења (енг. \textit{Electronic Data Gathering, Analysis, and Retrieval} - EDGAR) је примарна база података Комисије за хартије од вредности Сједињених Америчких Држава (SEC) за финансијске пријаве. Све јавне компаније дужне су да своје изјаве, периодичне извештаје и друге обрасце за обелодањивање података подносе путем EDGAR-а, уместо папирних пријава. EDGAR ове пријаве чини бесплатно доступним јавности, омогућавајући инвеститорима да преузимају и претражују хиљаде докумената компанија и фондова. Поред корпоративних пријава, EDGAR такође садржи обелодањивања и других субјеката као што су узајамни фондови, берзански трговани фондови (ETF-ови), варијабилне ануитете и инсајдери, чиме се унапређује транспарентност америчких финансијских тржишта.

\subsection{Најчешће врсте извештаја доступних на EDGAR-у}

EDGAR организује пријаве према стандардизованим типовима образаца. Кључни извештаји доступни у бази EDGAR укључују:

\begin{itemize}
\item \textbf{Образац 10-K (енг. \textit{Form 10-K}) (Годишњи извештај):} Свеобухватан годишњи извештај који америчке јавне компаније морају да поднесу након завршетка сваке фискалне године. Форма 10-K садржи ревидиране годишње финансијске извештаје, дискусију о пословању компаније и материјалним факторима ризика, као и менаџментову дискусију и анализу финансијских резултата за годину.

\item \textbf{Образац 10-Q (енг. \textit{Form 10-Q}) (Квартални извештај):} Краћи извештај који се подноси за фискални квартала. Форма 10-Q укључује кварталне финансијске извештаје, ажурирања о свим значајним променама или материјалним ризицима од последњег 10-K/10-Q.

\item \textbf{Образац 8-K (енг. \textit{Form 8-K}) (Текући извештај):} Извештај који се подноси ради обелодањивања великих корпоративних догађаја у реалном времену, уместо чекања на наредни 10-Q или 10-K. Компаније подносе 8-K кад год се догоде значајни догађаји које акционари треба да знају.
\end{itemize}

\section{Преглед форме годишњег извештаја 10-K}

Образац 10-K је годишњи извештај који прописује SEC и који америчке јавне компаније подносе, а нуди детаљну слику финансијског стања и пословних активности компаније током претходне године. 10-K укључује темељан опис пословних операција компаније, дискусију о ризицима са којима се компанија суочава, и ревидиране финансијске резултате за фискалну годину. Менаџмент компаније такође пружа анализу и контекст за финансијске резултате, објашњавајући покретаче перформанси и све трендове или неизвесности који би могли утицати на будуће резултате.

\subsection{Садржај и структура извештаја Форма 10-K}

Форма 10-K је подељена на четири главна дела, при чему сваки део садржи неколико ставки како је прописано регулативом SEC-а. Укупно, постоји 15 нумерисаних ставки које морају бити обухваћене у 10-K. У наставку је преглед ових делова и информација које сваки садржи:

\subsubsection{Део I – Преглед пословања и ризика}

\begin{itemize}
\item \textbf{Ставка 1. Пословање (енг. \textit{Item 1. Business}):} опис операција, главни производи/услуге, зависна друштва, тржишта, конкуренција, регулатива, сезоналност.

\item \textbf{Ставка 1А. Фактори ризика (енг. \textit{Item 1A. Risk Factors}):} најзначајнији ризици по компанију/хартије; набројани по значају.

\item \textbf{Ставка 1Б. Нерешени коментари особља (енг. \textit{Item 1B. Unresolved Staff Comments}):} материјални нерешени SEC коментари (ако постоје).

\item \textbf{Ставка 2. Непокретности (енг. \textit{Item 2. Properties}):} значајна физичка имовина (погони, капацитети, рудници, канцеларије, некретнине).

\item \textbf{Ставка 3. Судски спорови (енг. \textit{Item 3. Legal Proceedings}):} нерутинске парнице/регулаторни поступци.

\item \textbf{Ставка 4. Обелодањивања о безбедности у рудницима (енг. \textit{Item 4. Mine Safety Disclosures}):} често „резервисано" (без садржаја).
\end{itemize}

\subsubsection{Део II – Финансијске информације и резултати}

\begin{itemize}
\item \textbf{Ставка 5. Тржиште обичних акција и питања акционара (енг. \textit{Item 5}):} берзе, број акционара, политика дивиденди, откупи/емисије.

\item \textbf{Ставка 6. Одабрани финансијски подаци (енг. \textit{Item 6. Selected Financial Data}):} кључни показатељи (обично 5 година) у сажетку.

\item \textbf{Ставка 7. MD\&A (енг. \textit{Item 7. Management's Discussion and Analysis}):} анализа стања/резултата, ликвидност, ресурси, трендови/неизвесности, кључне процене.

\item \textbf{Ставка 7А. Тржишни ризик (енг. \textit{Item 7A. Quantitative and Qualitative Disclosures About Market Risk}):} изложености (каматни, девизни, робни, цена капитала) и управљање/хеџинг.

\item \textbf{Ставка 8. Финансијски извештаји и додатни подаци (енг. \textit{Item 8}):} аудитирани FS (биланс стања, успеха, токови готовине, капитал) + напомене + мишљење ревизора.

\item \textbf{Ставка 9. Промене и неслагања са рачуновођама (енг. \textit{Item 9}):} промена ревизора и материјална неслагања (ако их је било).

\item \textbf{Ставка 9А. Контроле и процедуре (енг. \textit{Item 9A. Controls and Procedures}):} ефикасност контроле обелодањивања и ИКФИ; SOX 404/атестација.

\item \textbf{Ставка 9Б. Остале информације (енг. \textit{Item 9B. Other Information}):} информације које су требале у Форми 8-K у Q4, а нису објављене.
\end{itemize}

\subsubsection{Део III – Руководство и управљање}

\begin{itemize}
\item \textbf{Ставка 10. Директори, извршни руководиоци и корпоративно управљање (енг. \textit{Item 10}):} биографије, кодекс етике, структура одбора/чланства (често упућивање на прокси).

\item \textbf{Ставка 11. Накнаде извршних руководилаца (енг. \textit{Item 11. Executive Compensation}):} плате, бонуси, акцијске награде, политике програма.

\item \textbf{Ставка 12. Власништво и питања акционара (енг. \textit{Item 12. Security Ownership}):} власништво >5\%, руководиоци/директори; планови капиталних компензација.

\item \textbf{Ставка 13. Повезана лица и независност директора (енг. \textit{Item 13. Certain Relationships}):} материјалне трансакције са инсајдерима; независност директора.

\item \textbf{Ставка 14. Накнаде и услуге главног рачуновође (енг. \textit{Item 14. Principal Accountant Fees and Services}):} накнаде ревизорској фирми по врстама услуга.
\end{itemize}

\subsubsection{Део IV – Прилози и финансијски распореди}

\begin{itemize}
\item \textbf{Ставка 15. Прилози, финансијски распореди (енг. \textit{Item 15. Exhibits, Financial Statement Schedules}):} списак свих прилога/распореда (оснивачки акти, статут, значајни уговори, списак зависних, сертификати CEO/CFO — Exhibit 31/32); приступ преко EDGAR-а.
\end{itemize}

Сваки од ових делова и ставки организован је на конзистентан начин за све компаније, што олакшава навигацију кроз 10-K када се упознате са његовом структуром.

\section{Скуп података за тестирање - FinanceBench}

FinanceBench је референтни тест за одговарање на финансијска питања. Обухвата укупно 10.231 пар питања и одговора о јавним компанијама у САД. Питања су утемељена на стварним корпоративним извештајима из EDGAR-а описаним у секцији \ref{sec:3}. Укупно, скуп података покрива 40 компанија из САД у различитим индустријским секторима. Свако питање прати верификован одговор и подржавајући исечак доказа који оправдава одговор. Лабелари и финансијски стручњаци, осигурали су да свако питање буде јасно и једноставно за одговор на основу поднесака, тако да бенчмарк поставља минимални стандард перформанси за тачност модела. Табела \ref{tab:financebench} сумира кључне карактеристике скупа података FinanceBench \cite{islam_financebench_2023}.

\begin{table}[h]
\centering
\begin{tabular}{|p{4cm}|p{10cm}|}
\hline
\textbf{Карактеристика} & \textbf{Опис} \\
\hline
Укупно парова П-О (QA) & 10.231 питање о јавним компанијама, свако са златним одговором и пратећим доказом \cite{islam_financebench_2023}. \\
\hline
Обухваћене компаније & 40 јавно тргованих компанија (САД), у више индустријских сектора (ГИКС — енг. \textit{Global Industry Classification Standard}, GICS). \\
\hline
Изворни документи & 361 финансијски извештај (нпр. 10-K, 10-Q, 8-K, транскрипти позива о заради) из периода 2015–2023. \\
\hline
Подскуп отвореног кода & 150 П-О инстанци (евалуациони узорак) јавно објављених са анотацијама. \\
\hline
Поља по уносу & Питање (финансијски упит), Одговор (анотирани тачан одговор), Доказ (одломак из поднеска компаније са референтном страницом), Образложење (опционално објашњење начина закључивања). \\
\hline
Категорије питања & 3 типа: Доменски релевантна (енг. \textit{Domain-Relevant}; општа аналитичка питања), Ново-генерисана (енг. \textit{Novel-Generated}; експертски креирана, специфична за компанију), Метрички генерисана (енг. \textit{Metrics-Generated}; питања о финансијским метрикама). \\
\hline
Потребно расуђивање & Означено типом расуђивања: $\sim$28\% чиста информациона екстракција, 66\% укључује нумеричке прорачуне, 6\% логичко/аналитичко расуђивање. \\
\hline
\end{tabular}
\caption{Карактеристике скупа података FinanceBench}
\label{tab:financebench}
\end{table}

\subsection{Типови питања и композиција}

Питања у FinanceBench су категоризована у три групе које одражавају начин на који су генерисана и вештине потребне за одговор:

\begin{enumerate}
\item \textbf{Доменски релевантна питања:} Фиксни скуп од 25 питања која су широко примењива у анализи било које јавне компаније. Она укључују питања које би поставио финансијски аналитичар, као што су да ли је компанија исплатила дивиденду у последњој години или да ли су оперативне марже остале стабилне током времена.

\item \textbf{Ново-генерисана питања:} Ово су оригинална питања која су писали финансијски аналитичари, специфична за контекст сваке компаније, садржај извештаја и индустрију. Нова питања покривају разнолике теме (нпр. пословну стратегију компаније, значајне догађаје или необичне ставке), и формулисана су на различите начине да имитирају упите из стварне праксе. Овај подскуп је примењен на 37 компанија, при чему свака има приближно између 15 и 80 прилагођених питања (просечно 36), укупно 1.323 П-О инстанце.

\item \textbf{Програмски генерисана питања:} Убедљиво највећи део чине питања аутоматски изведена из финансијских метрика. Анотатори су најпре извукли $\sim$18 фундаменталних финансијских метрика (нпр. приход, нето добит, различите билансне и новчане ставке) из извештаја сваке компаније током 8 година (2015–2022). Полазећи од њих, скрипта је генерисала бројна питања о базним метрикама и различитим изведеним метрикама рачунатим из њих. На пример, шаблони би формирали питања као што су „Колика је бруто маржа компаније X у 2021?" или „Који је однос амортизације (из извештаја о токовима готовине) према укупном приходу за 2021?", где се одговор може израчунати из пријављених бројки. Нека од ових питања су чисто екстрактивна (траже једну наведну цифру), док друга подразумевају вишестепену аритметику или комбиновање података из више извештаја \cite{islam_financebench_2023}. Метричка генерација је примењена на 32 компаније, производећи између 135 и 348 питања по компанији ($\approx$249 у просеку), укупно 7.983 пара \cite{islam_financebench_2023}. Ова категорија тестира способност модела да спроведе нумеричко резоновање.
\end{enumerate}